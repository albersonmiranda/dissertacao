% elementos pré-textuais 

% título do sumário
\ifdefined\contentsname
  \renewcommand*\contentsname{SUMÁRIO}
\else
  \newcommand\contentsname{SUMÁRIO}
\fi

% capa 
\imprimircapa

% folha de rosto 
% o * indica que haverá a ficha bibliográfica 
\imprimirfolhaderosto*

% ficha catalográfica 
%\begin{fichacatalografica}
%    \includepdf{ficha_ufes.pdf}
%\end{fichacatalografica}


% substituir pela ficha em pdf fornecida pela UFES após defesa 
\begin{fichacatalografica}
	\sffamily
	\vspace*{\fill}					% Posição vertical
	\begin{center}					% Minipage Centralizado
	\fbox{\begin{minipage}[c][8cm]{15cm}		% Largura
	\small
	\imprimirautor
	
	\hspace{0.5cm} \imprimirtitulo  / \imprimirautor. --
	\imprimirlocal, \imprimirdata-
	
	\hspace{0.5cm} \thelastpage p. : il. (algumas color.) ; 30 cm.\\
	
	\hspace{0.5cm} \imprimirorientadorRotulo~\imprimirorientador\\
	
	\hspace{0.5cm}
	\parbox[t]{\textwidth}{\imprimirtipotrabalho~--~\imprimirinstituicao,
	\imprimirdata.}\\
	
	\hspace{0.5cm}
		1. Economia Bancária.
		2. Séries Temporais Hierárquicas.
		3. Reconciliação Ótima.
    4. Machine Learning.
		I. Pereira, Guilherme Armando de Almeida.
		II. Universidade Federal do Espírito Santo.
		III. Centro de Ciências Jurídicas e Econômicas.
		IV. Título 			
	\end{minipage}}
	\end{center}
\end{fichacatalografica}

% folha de aprovação 
%
%\begin{folhadeaprovacao}
%    \includepdf{folhadeaprovacao_final.pdf}
%\end{folhadeaprovacao}


% substituir pela folha assinada pela banca após defesa 
\begin{folhadeaprovacao}

  \begin{center}
    {\ABNTEXchapterfont\large\imprimirautor}

    \vspace*{\fill}\vspace*{\fill}
    \begin{center}
      \ABNTEXchapterfont\bfseries\Large\imprimirtitulo
    \end{center}
    \vspace*{\fill}
    
    \hspace{.45\textwidth}
    \begin{minipage}{.5\textwidth}
        \imprimirpreambulo
        \vspace*{1cm}
        Aprovada em 29 de fevereiro de 2024.\\[2cm]
        \textbf{COMISSÃO EXAMINADORA} \\
        \assinatura{\textbf{\imprimirorientador} \\ Universidade Federal do Espírito Santo \\ Orientador} 
        \assinatura{\textbf{Prof. Dr. Edson Zambon Monte} \\ Universidade Federal do Espírito Santo}
        \assinatura{\textbf{Prof. Dr. Fernando Luiz Cyrino Oliveira} \\ PUC-Rio}
        %\assinatura{\textbf{Professor} \\ Convidado 3}
        %\assinatura{\textbf{Professor} \\ Convidado 4}
    \end{minipage}%
   \end{center}
  
\end{folhadeaprovacao}

%% dedicatória 
%\begin{dedicatoria}
%   \vspace*{\fill}
%   \centering
%   \noindent
%   \textit{Exemplo de dedicatória,\\\lipsum[10].} \vspace*{\fill}
%\end{dedicatoria}

%% agradecimentos 
\begin{agradecimentos}
  A jornada para conclusão do mestrado não poderia ter sido tão divertida e gratificante sem a influência de várias pessoas. Primeiramente, gostaria de agradecer ao meu orientador, Dr. Guilherme Pereira, por sua orientação, gentileza e constante encorajamento ao longo de todo o processo. Sinto como se tivesse ganhado um novo par de olhos que me permitem me ver e ao mundo de uma maneira diferente.

  Ao corpo docente do PPGEco/UFES, em especial meus professores Edson Zambon, por sua meticulosidade e Alain Herscovici, pelas inúmeras conversas que me permitiram compreender celebrar a cacofonia de ideias, da música à filosofia, da sociologia à meta-ciência, me permitindo comrpeender que não há uma única maneira de se fazer ciência.

  Ao professor Nikolaos Kourentzes pela resposta imediata sobre o dataset \textit{visitor nights}. À Evangelos Spiliotis, cuja contribuição à literatura inspirou este trabalho, por seus esclarecimentos acerca de seu artigo e por ter compartilhado seu código.
  
  Dentre várias pessoas que me ajudaram [...] Apesar de eu me dizer que consigo fazer qualquer coisa, às vezes eu preciso de ajuda para me convencer.

  Aos meus professores do departamento de matemática do Ifes/Vitória, em especial aos professores Diogo Oliveira, pelas inúmeras conversas e encorajamento; à professora Débora [..]; ao professor Lourenço, que me catapultaram ; Me sinto mais forte, mais capaz e com vontade de encarar de frente novos desafios. Principalmente aqueles que eu não sei se consigo resolver.

  Aos meus colegas do programa, Dreyfuss, Yasmin, Amanda, Leina, Eduardo, Ricardo 

  À minha família: minha companheira Rafaela, minha mãe Regina, meu pai Aldo e minhas sobrinhas. Ter cursado o mestrado enquanto trabalhava em tempo integral e cursava uma segunda graduação em matemática à noite pode não ter sido mais senstata, mas vocês não deixaram eu me questionar em nenhum momento.
\end{agradecimentos}

% epígrafe 
%\begin{epigrafe}
%    \vspace*{\fill}
%	\begin{flushright}
%		\textit{``Modelo de epígrafe, \\
%		modelo de epígrafe.''}
%	\end{flushright}
%\end{epigrafe}

% resumo 

\setlength{\absparsep}{18pt}
\begin{resumo}
  Na última década, a previsão de séries temporais hierárquicas experimentou um crescimento substancial, caracterizado por avanços que melhoraram significativamente a precisão dos modelos de previsão. Recentemente, os métodos de \textit{machine learning} foram integrados à literatura de previsão hierárquica como uma nova abordagem para a reconciliação de previsões. Este trabalho se baseia nesses avanços, explorando ainda mais o potencial dos métodos de \textit{machine learning} para otimizar a reconciliação de séries temporais hierárquicas e agrupadas. Além disso, investigamos o impacto de várias estratégias de aquisição de conjuntos de treinamento, como previsões obtidas por \textit{rolling forecasting}, valores ajustados de modelos reestimados e valores ajustados dos modelos de previsão base. Para avaliar a metodologia proposta, dois estudos de caso foram realizados. O primeiro estudo se concentra no setor financeiro brasileiro, especificamente na previsão de saldos de empréstimos e financiamentos para o Banco do Estado do Espírito Santo. O segundo estudo usa conjuntos de dados de turismo doméstico australiano, que são frequentemente referenciados na literatura de séries temporais hierárquicas. Comparamos nossa metodologia proposta com métodos analíticos para reconciliação de previsões, como o \textit{bottom-up}, \textit{top-down} e traço mínimo. Os resultados mostram que não há um método ou estratégia única que supere consistentemente todos os outros. No entanto, a combinação apropriada de método ML e estratégia pode levar a uma melhoria de até 93\% na precisão em comparação com o melhor método de reconciliação analítica.

  \textbf{Palavras-chave}: Séries Temporais Hierárquicas. Reconciliação Ótima. \textit{Machine Learning}. Economia Bancária.
\end{resumo}

% abstract 
\begin{resumo}[Abstract]
  \begin{otherlanguage*}{english}
    
    In the last decade, hierarchical time series forecasting has experienced substantial growth, characterized by advancements that have significantly improved the accuracy of forecasting models. Recently, machine learning methods have been integrated into the literature on hierarchical time series as a new approach for forecasting reconciliation.This paper builds upon these advancements by further exploring the potential of ML methods for optimizing the reconciliation of hierarchical and grouped time series. Moreover, we investigate the impact of various training set acquisition strategies, such as in-sample forecasts obtained through rolling origin forecasting, fitted values of reestimated models, and fitted values of base forecast models. To evaluate the proposed methodology, two case studies were carried out. The first study focuses on the Brazilian financial sector, specifically forecasting loan and financing balances for the State Bank of Espírito Santo. The second study uses Australian domestic tourism datasets, which are frequently referenced in hierarchical time series literature. We compared our proposed methodology with traditional methods for forecasting reconciliation such as bottom-up, top-down and minimum trace.The results show that there is no unique method or strategy that consistently outperforms all others. Nonetheless, the appropriate combination of ML method and strategy can lead to up to a 93\% improvement in accuracy compared to the best-performing analytical reconciliation method.
    \vspace{\onelineskip}
 
    \noindent 
    \textbf{Keywords}: Hierarchical Time-Series. Optimal Reconciliation. Machine Learning. Economics of Banking.
  \end{otherlanguage*}
\end{resumo}

% lista de ilustrações 
\pdfbookmark[0]{\listfigurename}{lof}
\listoffigures*
\cleardoublepage

% lista de quadros 
%\pdfbookmark[0]{\listofquadrosname}{loq}
%\listofquadros*
%\cleardoublepage

% lista de tabelas 
\pdfbookmark[0]{\listtablename}{lot}
\listoftables*
\cleardoublepage

% lista de abreviaturas 
\begin{siglas}
  \item[Banestes] BANco do ESTado do Espírito Santo
  \item[ETS] \textit{Exponentional Smoothing}
  \item[Favar] \textit{Factor Augmented Vector Autoregression}
  \item[Lasso] \textit{Least Absolute Shrinkage and Selection Operator}
  \item[LGBM] \textit{Light Gradient Boosting Machine}
  \item[MCRL] Modelo Clássico de Regressão Linear
  \item[MASE] \textit{Mean Absolute Scaled Error}
  \item[MinT] \textit{Minimum Trace}
  \item[ML] \textit{Machine Learning}
  \item[MQGF] Mínimos Quadrados Generalizados Factíveis
  \item[MQO] Mínimos Quadrados Ordinários
  \item[MQP] Mínimos Quadrados Ponderados
  \item[PIB] Produto Interno Bruto
  \item[RF] \textit{Random Forest}
  \item[RMSSE] \textit{Root Mean Squared Scaled Error}
  \item[SFN] Sistema Financeiro Nacional
  \item[SVM] \textit{Support Vector Machines}
  \item[SVR] \textit{Support Vector Regression}
  \item[XGBoost] \textit{Extreme Gradient Boosting}
\end{siglas}

% lista de símbolos 
\begin{simbolos}
  \item[$ t $] Tempo dentro da amostra
  \item[$ T $] Último tempo dentro da amostra
  \item[$ h $] Horizonte de previsão, tempo fora da amostra
  \item[$ \Omega $] Conjunto de dados dentro da amostra
  \item[$ y $] Série temporal dentro da amostra
  \item[$ \hat{y} $] Série temporal estimada
  \item[$ \tilde{y} $] Série temporal reconciliada
  \item[$ n $] Número de séries na hierarquia
  \item[$ m $] Número de séries no menor nível da hierarquia
  \item[$ k $] Número de níveis na hierarquia
  \item[$ \mathbfit{S} $] Matriz de soma
  \item[$ \mathbfit{G} $] Matriz de reconciliação
  \item[$ \{...\} $] Conjunto
  \item[$ |\{...\}| $] Cardinalidade de um conjunto
\end{simbolos}

% sumário 
\pdfbookmark[0]{\contentsname}{toc}
\tableofcontents*
\cleardoublepage

% elementos textuais 
\textual
\pagestyle{simple}