% elementos pós-textuais 
\postextual

%% apêndice 
%\begin{apendicesenv}
%\partapendices
%
%\chapter{PRIMEIRO APÊNDICE}
%
%\lipsum[50]
%
%\chapter{SEGUNDO APÊNDICE}
%
%\lipsum[50]
%
%\end{apendicesenv}

% anexos 
\begin{anexosenv}
\partanexos

\chapter{código para construção da base de dados}\label{anexo_a}

\begin{lstlisting}[frame=single]
# pacotes
library(magrittr, include.only = "%>%")

# municípios x regiões imediatas
municipios = basedosdados::read_sql("
SELECT
  id_municipio
  , nome
  , nome_mesorregiao
  , nome_microrregiao
  , nome_regiao
  , nome_regiao_imediata
  , nome_regiao_intermediaria
FROM `basedosdados.br_bd_diretorios_brasil.municipio`
WHERE sigla_uf = 'ES'
")

# importando dados
estban = basedosdados::read_sql("
SELECT
  CAST(ano AS STRING) AS ano
  , CAST(mes AS STRING) AS mes
  , id_municipio
  , cnpj_agencia
  , CASE
        WHEN id_verbete = '160' THEN 'operações de crédito'
        WHEN id_verbete = '161' THEN 'empréstimos e títulos descontados'
        WHEN id_verbete = '162' THEN 'financiamentos'
        WHEN id_verbete = '163' THEN 'financiamentos rurais'
        WHEN id_verbete = '169' THEN 'financiamentos imobiliários'
        WHEN id_verbete = '172' THEN 'outros créditos'
        WHEN id_verbete = '174' THEN 'provisão para operações de crédito'
        ELSE 'outros'
    END AS verbete
  , valor

FROM `basedosdados.br_bcb_estban.agencia`

WHERE
  cnpj_basico = '28127603'
  AND id_verbete IN ('161', '162', '163', '169')
")

# formatando datas
estban = within(estban, {
  mes = formatC(as.numeric(mes), format = "d", width = 2, flag = "0")
  ref = as.Date(paste(ano, mes, "01", sep = "-"))
})

# identificando agências em atividade
agencias_fim = subset(estban, ref == max(ref), select = cnpj_agencia) |>
  (\(x) unique(x$cnpj_agencia))()

# filtrando apenas agências em atividade
estban = subset(estban, cnpj_agencia %in% agencias_fim)

# mesclando com tabela municípios
estban_df = merge(estban, municipios, by = "id_municipio")

# adicionando estrutura hierárquica e agrupada
estban = tsibble::tsibble(
  estban_df,
  index = ref,
  key = c("id_municipio", "cnpj_agencia", "verbete", "nome_mesorregiao")
)

estban = estban |>
  fabletools::aggregate_key(
    (nome_mesorregiao / id_municipio / cnpj_agencia) * verbete,
    valor = sum(valor)
)
\end{lstlisting}

\end{anexosenv}

% índice remissivo 
\phantompart
\printindex
